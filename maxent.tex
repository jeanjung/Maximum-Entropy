\documentclass[a4wide]{article}
\usepackage{amsmath,amssymb}
\usepackage{stmaryrd}
\usepackage{latexsym}

\title{Maximum Entropy -- Lessons learnt}
\date{\today}
\author{Jean Christoph Jung}

\newcommand{\infers}{\mathrel|\joinrel\sim}
\newcommand{\lsem}{[\![}
\newcommand{\rsem}{]\!]}

\begin{document}
\maketitle

\section{Representation Dependence}

\subsection{Representation Dependence in Probabilistic Inference \cite{HalpernK04}}

The article \cite{HalpernK04} considers the notion of ``representation dependence'' which
is a very common notion in non-deductive (inductive) reasoning systems: The results of an
inference procedure depend heavily of the chosen representation formalism, \mbox{i.e.}, the
way the knowledge is written down. In particular representation dependence is one of the
major points of criticism of the principle of maximum entropy. 

The following setting is considered: 
\begin{itemize}
  \item some state space $X$ 
  \item Given a knowledge base $KB$, $\lsem KB\rsem_X$ is the set of probability distributions on $X$
    that is allowed by $KB$.
  \item \emph{entailment}: $KB\models\theta$ if $\lsem KB\rsem\subseteq \lsem\theta\rsem$.
  \item $X$-inference procedure is a monotonic decreasing function $I_X$. We write $KB\infers_I\theta$ if 
    $I_X(\lsem KB\rsem )\subseteq \lsem\theta\rsem$
  \item \emph{embedding}: function $f:X\rightarrow Y$ that maps one state space to another with the only
    condition that $f$ is a homomorphism with respect to conjunction and negation.
  \item $f$ induces $f^*$: $f^*(\mu)=\{\nu\in\Delta_Y\mid \nu(f(S))=\mu(S)\mathrm{\ for\ all\ } S\subseteq X\}$
  \item $I$ is \emph{invariant} under $f$ if for all $KB$ and $\theta$, we have $KB\infers_I\theta$ iff
    $f^*(KB)\infers_I f^*(\theta)$.
\end{itemize}

Inuitively, embeddings are representation shifts. Obviously, not every embedding is
an appropriate representation shift, \mbox{e.g.}, $f(p)=f(q)$ gives us that $f(p\wedge\neg q)$
becomes unsatisfiable although $p\wedge\neg q$ was satisfiable. To prevent such situations,
an embedding is called \emph{faithful} if $f(S)\neq\emptyset$ for all $S\subseteq X$.
In the following they restrict their attention to faithful embeddings and call an inference relation
$\infers_I$ \emph{representation independent} if it is invariant under all faithful embeddings. 


They show that representation independence definied in this way is quite a strong notion. First, not
many inference procedures have this property. Second, we can not add further restriction without
getting only not interesting inference procedures or none at all. (The two examples in 
the paper are ``kind of'' \emph{irrelevant information} and \emph{minimal default independence}.)




\section{From Statistical Information to Beliefs \cite{BacchusGHK96}}




\bibliographystyle{plain}
\bibliography{maxent}


\end{document}
